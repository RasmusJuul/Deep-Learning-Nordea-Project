\documentclass{article}

\usepackage{amsmath}
\usepackage{graphicx}
\usepackage{float}
\usepackage{amssymb}
\usepackage{blindtext}



\title{Financial modelling\\ Using public data to predict companies revenues}
\author{Rasmus P. s164564, Luka A. s191963, Christian P. s124267}
\date{\today}



\begin{document}
\maketitle
\section*{Motivation}
Deep learning is to an increasing degree being used within all fields where data is a crucial part of gaining an advantage over your competitors. The financial business sector is a sector where deep learning can be used to a great extent, with potentially very high reward. 
The potential gains by being able to create a neural network that can predict the revenue of a company somewhere in the future could make for better abilities in issuing loans, potential changes in company value etc.

\section*{Background}
Creating and optimizing a neural network to succesfully forecast the revenue of a company given publicly accessible data. The forecast should work with a margin(confidence interval) of the forecast.

\section*{Milestones}
\begin{enumerate}
	\item Optain and analyse data
	\item Design appropriate network architecture (currently looking at MDN as an option, maybe combined with an RNN)
	\item Optimize on the network to achieve good predictions.
	\item Report the findings.
\end{enumerate}
\section*{References}
https://towardsdatascience.com/predicting-probability-distributions-using-neural-networks-abef7db10eac

\end{document}
